\documentclass[a4paper,10pt]{report}
\usepackage[latin1]{inputenc}
\usepackage{html}
\usepackage[nofancy]{svninfo}
\usepackage{listings}

% Title Page
\title{JADE Priority-Based Scheduler Add-On}
\author{Juan A. Su�rez Romero (jsuarezr@udc.es)%
	\and Amparo Alonso Betanzos (ciamparo@udc.es)%
	\and Bertha Guijarro Berdi�as (cibertha@udc.es)}

\lstset{language=Java, frame=trBL, breaklines=true}

\sloppy

\begin{htmlonly}
  \usepackage{verbatim}
  \providecommand{\lstinputlisting}[2][]{\verbatiminput{#2}}
  \providecommand{\svnInfoLongDate}{\verb+$Date$+}
\end{htmlonly}

\begin{document}
\maketitle

% \begin{abstract}
% \end{abstract}

\section*{Requirements}
This add-on has been tested with \emph{Sun JDK 1.4} and \emph{JADE 3.4}.

\section*{Description}
In the current version of \emph{JADE}\footnote{At this moment the current
version of JADE is 3.4.} an agent can contain several behaviours. Each agent
contains an internal scheduler that manages its behaviuors in a round-robin
manner. That is, the scheduler selects the first created behaviour and executes
it\footnote{Executing a behaviour means calling the action() method of the
behaviour.}. Then it selects the next one, and also executes it, and so on. When
the last behaviour is executed, the scheduler restart again with the first
behaviour, and so forth.

The user are not able to assign more priority to one behaviour than to others:
all behaviours are executed with the same priority. In some situations, as can
be seen in several posts to the \emph{JADE mail list}, it would be desirable to
assign different priorities to each behaviour so the scheduler executes more
frequently those behaviours with a higher priority.

The aim of this add-on is to provide a scheduler to \emph{JADE} that is able to
manage the behaviours using priorities. As it can be seen soon, this is done by
providing two new \texttt{CompositeBehaviours} in which the children are
scheduled using priorities. One of these behaviuors schedule its children in a
parallel way, while the other behaviour schedule its children in a sequential
way.

\subsection*{Concurrent management of behaviours with priorities}
This add-on provides a new behaviour named \texttt{ParallelPriorityBehaviour}
that schedules its children in a parallel manner. This behaviour is similar to
\texttt{ParallelBehaviour}, but in this case using priorities to select the
executing behaviours. So those behaviours with a higher priority are executed
more
frequently than other behaviours with a lesser priority.

Each child is assigned by the user with a priority equal or greater than
\texttt{0}. The greater the number, the lesser the priority. So a child with
priority value of \texttt{3} has more priority than other behaviour with
priority value of \texttt{5}. The highest priority a behaviour can reach is
\texttt{0}. This priority assigned by the user is called \emph{static priority},
and can be changed by the user at any moment.

Internally, the \texttt{ParallelPriorityBehaviour}'s scheduler uses another
priority that is called \emph{dynamic priority}. As its name indicates, this
priority is increased dynamically by the scheduler in order to avoid the
starvation of a behaviour. So the dynamic behaviour of a child is a number
between the static priority value of this child and \texttt{0}.

The idea of the algorithm is that the scheduler executes those children with a
dynamic priority value of \texttt{0}. If there are several behaviours that have
dynamic priority of \texttt{0} then they are executed in the order in which they
were inserted. Also, if the selected behaviour is blocked then the scheduler
does not execute it. At last, if there are not children with these dynamic
priority of \texttt{0} then all dynamic priorities are increased.

In order to understand better the functioning of the algorithm, we will use an
example. We will create a \texttt{ParallelPriorityBehaviour} with four children
$A$, $B$, $C$ and $D$. Each child is assigned with (static) priority of
\texttt{4}, \texttt{1}, \texttt{0} and \texttt{1}, respectively. These children
are cyclic behaviours that ends when they are executed twice (see listing
\ref{code:children}). The \texttt{ParallelPriorityBehaviour} ends when all its
children end (see listing \ref{code:agentparallel}).

\begin{lstlisting}[caption={Source code for children}, label=code:children]
import jade.core.behaviours.Behaviour;

public class ChildBehaviour extends Behaviour {
  int counter=2;
  String id;
 
  public ChildBehaviour(String id) {
    super();
    this.id=id;
  }

  public void action() {
    System.out.print (id+"->");
    counter--;
  }

  public boolean done() {
    return counter==0;
  }
}
\end{lstlisting}

\begin{lstlisting}[caption={Source code for the agent using
\texttt{ParallelPriorityBehaviour}}, label=code:agentparallel]
import jade.core.Agent;
import jade.core.behaviours.ParallelPriorityBehaviour;

public class ParallelAgent extends Agent {
  protected void setup() {
    ParallelPriorityBehaviour ppb = new ParallelPriorityBehaviour();
    this.addBehaviour(ppb);
    ppb.addSubBehaviour(new ChildBehaviour("A"), 4);
    ppb.addSubBehaviour(new ChildBehaviour("B"), 1);
    ppb.addSubBehaviour(new ChildBehaviour("C"), 0);
    ppb.addSubBehaviour(new ChildBehaviour("D"), 1);
  }
}
\end{lstlisting}


Below we show the steps followed by the scheduler. These steps are also
summarized in table \ref{table:parallelprioritybehaviour}. In this table the
superindex of each child is its static priority, while the subindex is its
dynamic priority.

\begin{enumerate}
\item At first the dynamic priority of each child is assigned to the static
priority value.
\item The scheduler selects those sub-behaviours that have dynamic priority of
\texttt{0}, and executes it. In this case the selected sub-behaviour is $C$.
\item Again the scheduler selects those sub-behaviours with dynamic priority of
\texttt{0}, that is $C$ once again.
\item Because $C$ has been executed twice, it ends. So the scheduler removes it.
\item The scheduler tries to select the children with dynamic priority of
\texttt{0}. But now there are not such sub-behaviours, so the scheduler
increases the dynamic priority of all sub-behaviours step by step, until one or
more of them reaches a value \texttt{0} in its dynamic priority.
\item Now that there are several sub-behaviours with dynamic priority of
\texttt{0}, the scheduler executes the child that was inserted first ($B$).
\item After executing once the child $B$, its dynamic priority is reset to its
static priority.
\item Again the scheduler selects the sub-behaviour that has dynamic priority of
\texttt{0}, that is $D$, and executes it.
\item Once $D$ is executed, its dynamic priority is reset.
\item Again, because there are not children with dynamic priority of \texttt{0},
all dynamic priorities are increased.
\item The scheduler selects the first inserted behaviour with dynamic priority
of \texttt{0}, and executes it (that is, $B$ again).
\item Because this sub-behaviour has been executed twice, it ends, and the
scheduler drops it.
\item The scheduler selects the next child with dynamic priority of \texttt{0},
that is $D$.
\item Also this sub-behaviour ends because it has been executed twice. Thus the
scheduler removes it.
\item Because the remaining child does not have a dynamic priority of
\texttt{0}, this priority is increased.
\item Now the scheduler can execute the sub-behaviour $A$.
\item After this, the dynamic priority value is reset.
\item Once again, the dynamic priority value is increased until it reaches
\texttt{0}.
\item The scheduler executes $A$.
\item As $A$ was executed twice, it is removed, and the
\texttt{ParallelPriorityBehaviour} ends.
\end{enumerate}

\begin{table}
\centering
\begin{tabular}{|r|l|}
\cline{1-2}
STEP & CHILDREN \\
\cline{1-2}
1 & $\langle A_4^4,\:B_1^1,\:C_0^0,\:D_1^1\rangle$  \\ 
2 & $\langle A_4^4,\:B_1^1,\:\underline{C_0^0},\:D_1^1\rangle$  \\ 
3 & $\langle A_4^4,\:B_1^1,\:\underline{C_0^0},\:D_1^1\rangle$  \\ 
4 & $\langle A_4^4,\:B_1^1,\:D_1^1\rangle$  \\ 
5 & $\langle A_3^4,\:B_0^1,\:D_0^1\rangle$  \\ 
6 & $\langle A_3^4,\:\underline{B_0^1},\:D_0^1\rangle$  \\ 
7 & $\langle A_3^4,\:B_1^1,\:D_0^1\rangle$  \\ 
8 & $\langle A_3^4,\:B_1^1,\:\underline{D_0^1}\rangle$  \\
9 & $\langle A_3^4,\:B_1^1,\:D_1^1\rangle$  \\
10 & $\langle A_2^4,\:B_0^1,\:D_0^1\rangle$  \\
11 & $\langle A_2^4,\:\underline{B_0^1},\:D_0^1\rangle$  \\
12 & $\langle A_2^4,\:D_0^1\rangle$  \\
13 & $\langle A_2^4,\:\underline{D_0^1}\rangle$  \\
14 & $\langle A_2^4\rangle$  \\
15 & $\langle A_0^4\rangle$  \\
16 & $\langle \underline{A_0^4}\rangle$  \\
17 & $\langle A_4^4\rangle$  \\
18 & $\langle A_0^4\rangle$  \\
19 & $\langle \underline{A_0^4}\rangle$  \\
20 & $\langle \rangle$  \\
\cline{1-2}
\end{tabular}
\caption{Example of scheduling in \texttt{ParallelPriorityBehaviour}}
\label{table:parallelprioritybehaviour}
\end{table}

It is worth to note that even when the scheduler increases the dynamic priority
of behaviours, it does not in account whether these behaviours are blocked or
not. Only when a child is selected to be executed, the scheduler skips it if
this sub-behaviour is blocked.

Also, as in the case of the current \texttt{ParallelBehaviour}, if all children
are blocked then the \texttt{ParallelPriorityBehaviour} becomes blocked, until
one or more children becomes again runnable.

An interesting feature is that the new \texttt{ParallelPriorityBehaviour} can
replace the current \texttt{ParallelBehaviour}. If the user does not specify any
priority on children, the \texttt{ParallelPriorityBehaviour} behaves exactly
like the \texttt{ParallelBehaviour}.


\subsection*{Sequential management of behaviours with priorities}
Another composite behaviour provided by this add-on is
\texttt{SequentialPriorityBehaviour}. This new behaviour manages its
children in a sequential way, executing first those children with the highest
priority. In this way, this behaviour is similar to the current
\texttt{SequentialBehaviour}.

Each child is assigned with a priority by the user. The scheduler selects the
first inserted behaviour with the highest priority, and executes it until it
ends. During the execution, can happen several things. The first is that a new
child with a higher priority is inserted, or that an already existent child with
lesser priority changes to get a higher priority. In both cases, the scheduler
stops the execution of the current behaviour and it selects the new child,
starting to execute it until it ends.

The other issue happens when the selected child is blocked or becomes blocked.
If we were using a \texttt{SequentialBehaviour} this composite behaviour would
become blocked until the selected child becomes unblocked. In the case of the
new \texttt{SequentialPriorityBehaviour} this policy can be changed. The default
policy is the same as in the \texttt{SequentialBehaviour}, that is, the
\texttt{SequentialPriorityBehaviour} becomes blocked until the selected child
becomes runnable. The other policy is to skip the blocked children. So when the
\texttt{SequentialPriorityBehaviour} selects a blocked sub-behaviour, the
scheduler skips it and selects the next first inserted behaviour with the
highest priority that is runnable. In both cases if a new behaviour with a
higher priority appears, the scheduler selects the new one.

As in the last section, In order to understand the functioning of this new
behaviour, we will use an example similar to the one used in the last section.
In this case we will create a \texttt{SequentialPriorityBehaviour} using the
policy of skipping the blocked children (see listing
\ref{code:sequentialagent}). Also we will create four children, $A$, $B$, $C$
and $D$, with priority \texttt{4}, \texttt{1}, \texttt{0} and \texttt{1},
respectively. This sub-behaviours are the same as in the last section, and they
end when they are executed twice (see listing \ref{code:children}). To make the
example more complete, children $A$ and $B$ are blocked. In table
\ref{table:sequentialprioritybehaviour} can be noticed the steps followed by the
algorithm.

\begin{lstlisting}[caption={Source code for the agent using
\texttt{SequentialPriorityBehaviour}}, label=code:sequentialagent]
import jade.core.Agent;
import jade.core.behaviours.SequentialPriorityBehaviour;

public class SequentialAgent extends Agent {
  protected void setup() {
    SequentialPriorityBehaviour spb = new SequentialPriorityBehaviour();
    this.addBehaviour(spb);
    spb.addSubBehaviour(new ChildBehaviour("A"), 4);
    spb.addSubBehaviour(new ChildBehaviour("B"), 1);
    spb.addSubBehaviour(new ChildBehaviour("C"), 0);
    spb.addSubBehaviour(new ChildBehaviour("D"), 1);
  }
}
\end{lstlisting}

\begin{enumerate}
\item At first time, the scheduler searches for the highest priority behaviour
that has been inserted before. In this case, the child $C$ is selected, so the
scheduler executes it.
\item Because child $C$ does not still ends, the scheduler continues executing
it.
\item Now, as $C$ was executed twice, it ends. Therefore, the scheduler removes
$C$.
\item The scheduler searches again for the highest priority behaviour that has
been inserted first. In this case the selected behaviour is $B$. But as $B$ is
blocked and the policy is to skip the blocked children, the scheduler searches
for the next highest priority behaviour. And, in this case, the selected
behaviour is $D$, that is not blocked. So the scheduler executes it.
\item After the execution of $D$, we suppose that $A$ and $B$ becomes runnable.
Thus, the highest priority behaviour inserted first is $B$, and is runnable. So
the scheduler selects $B$ and executes it.
\item Again, $B$ is the highest priority behaviour, so the schedulers reruns it.
\item As $B$ ends, it is removed.
\item The scheduler searches for the new highest priority behaviour, which is
$D$. So the scheduler executes it.
\item As $D$ has been executed twice, it is removed from the
\texttt{SequentialPriorityBehaviour}.
\item Now the only remaining child is $A$, and because it is runnable, the
scheduler executes it.
\item As $A$ does not still end, it is executed once again.
\item $A$ ends, so it is removed. And because all children have finished, the
\texttt{SequentialPriorityBehaviour} ends.
\end{enumerate}


\begin{table}
\centering
\begin{tabular}{|r|l|}
\cline{1-2}
STEP & CHILDREN \\
\cline{1-2}
1 & $\langle A^4,\:B^1,\:\underline{C^0},\:D^1\rangle$  \\
2 & $\langle A^4,\:B^1,\:\underline{C^0},\:D^1\rangle$  \\
3 & $\langle A^4,\:B^1,\:D^1\rangle$  \\
4 & $\langle A^4,\:B^1,\:\underline{D^1}\rangle$  \\
5 & $\langle A^4,\:\underline{B^1},\:D^1\rangle$  \\
6 & $\langle A^4,\:\underline{B^1},\:D^1\rangle$  \\
7 & $\langle A^4,\:D^1\rangle$  \\
8 & $\langle A^4,\:\underline{D^1}\rangle$  \\
9 & $\langle A^4,\rangle$  \\
10 & $\langle \underline{A^4}\rangle$  \\
11 & $\langle \underline{A^4}\rangle$  \\
12 & $\langle \rangle$  \\
\cline{1-2}
\end{tabular}
\caption{Example of scheduling in \texttt{SequentialPriorityBehaviour}}
\label{table:sequentialprioritybehaviour}
\end{table}

It is worth to note that if all children are blocked, the
\texttt{SequentialPriorityBehaviour} becomes blocked.

Also, if we do not assign priorities and we use the default policy (that is, do
not skip the blocked sub-behaviours), the \texttt{SequentialPriorityBehaviour}
behaves exactly like the current \texttt{SequentialBehaviour}.

\section*{Credits}
Last update: \svnInfoLongDate

Authors:
\begin{itemize}
\item Juan Antonio Su�rez Romero
\item Amparo Alonso Betanzos
\item Bertha Guijarro Berdi�as
\end{itemize}

Laboratory for Research and Development in Artificial Intelligence (LIDIA).
Computer Science Department.
University of A Coru�a, Spain.

\vspace*{2em}

JADE - Java Agent DEvelopment Framework is a framework to develop multi-agent
systems in compliance with the FIPA specifications.
Copyright (C) 2000 CSELT S.p.A. 

GNU Lesser General Public License.

This library is free software; you can redistribute it and/or
modify it under the terms of the GNU Lesser General Public
License as published by the Free Software Foundation, 
version 2.1 of the License. 

This library is distributed in the hope that it will be useful,
but WITHOUT ANY WARRANTY; without even the implied warranty of
MERCHANTABILITY or FITNESS FOR A PARTICULAR PURPOSE.  See the GNU
Lesser General Public License for more details.

You should have received a copy of the GNU Lesser General Public
License along with this library; if not, write to the
Free Software Foundation, Inc., 59 Temple Place - Suite 330,
Boston, MA  02111-1307, USA.


\end{document}          
